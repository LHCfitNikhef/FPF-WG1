\documentclass[11pt,a4paper]{article}
\usepackage{placeins}
\usepackage{graphicx}
\usepackage{xcolor}
\usepackage{float}
\usepackage{afterpage}
\usepackage{amssymb,amsmath}
\usepackage{multirow,booktabs,multirow}
\usepackage{cite}
\usepackage[colorlinks=true, linkcolor=black!50!blue, urlcolor=blue, citecolor=blue, anchorcolor=blue]{hyperref}
\usepackage[font=small,labelfont=bf,margin=0mm,labelsep=period,tableposition=top]{caption}
\usepackage[a4paper,top=3cm,bottom=2.5cm,left=2.5cm,right=2.5cm,bindingoffset=0mm]{geometry}
\setlength{\unitlength}{1mm}

\usepackage{tabularx}
\newcolumntype{C}[1]{>{\centering\arraybackslash}p{#1}}

%%%%%%%%%%%%%%%%%%%%%%%%%%%%%%%%%%%%%%%%%%%%%%%%%%%%%%%%%%%%%

\def\smallfrac#1#2{\hbox{$\frac{#1}{#2}$}}
\newcommand{\be}{\begin{equation}}
	\newcommand{\ee}{\end{equation}}
\newcommand{\bea}{\begin{eqnarray}}
	\newcommand{\eea}{\end{eqnarray}}
\newcommand{\bi}{\begin{itemize}}
	\newcommand{\ei}{\end{itemize}}
\newcommand{\ben}{\begin{enumerate}}
	\newcommand{\een}{\end{enumerate}}
\newcommand{\la}{\left\langle}
\newcommand{\ra}{\right\rangle}
\newcommand{\lc}{\left[}
\newcommand{\rc}{\right]}
\newcommand{\lp}{\left(}
\newcommand{\rp}{\right)}
\newcommand{\as}{\alpha_s}
\newcommand{\aq}{\alpha_s\left( Q^2 \right)}
\newcommand{\amz}{\alpha_s\left( M_Z^2 \right)}
\newcommand{\aqq}{\alpha_s \left( Q^2_0 \right)}
\newcommand{\aqz}{\alpha_s \left( Q^2_0 \right)}
\newcommand{\Ord}{\mathcal{O}}
\newcommand{\MSbar}{\overline{\text{MS}}}
\def\toinf#1{\mathrel{\mathop{\sim}\limits_{\scriptscriptstyle
			{#1\rightarrow\infty }}}}
\def\tozero#1{\mathrel{\mathop{\sim}\limits_{\scriptscriptstyle
			{#1\rightarrow0 }}}}
\def\toone#1{\mathrel{\mathop{\sim}\limits_{\scriptscriptstyle
			{#1\rightarrow1 }}}}
\def\frac#1#2{{{#1}\over {#2}}}
\def\gsim{\gtrsim}
\def\lsim{\lesssim}
\newcommand{\mrexp}{\mathrm{exp}}
\newcommand{\dat}{\mathrm{dat}}
\newcommand{\one}{\mathrm{(1)}}
\newcommand{\two}{\mathrm{(2)}}
\newcommand{\art}{\mathrm{art}} 
\newcommand{\rep}{\mathrm{rep}}
\newcommand{\net}{\mathrm{net}}
\newcommand{\stopp}{\mathrm{stop}}
\newcommand{\sys}{\mathrm{sys}}
\newcommand{\stat}{\mathrm{stat}}
\newcommand{\diag}{\mathrm{diag}}
\newcommand{\pdf}{\mathrm{pdf}}
\newcommand{\tot}{\mathrm{tot}}
\newcommand{\minn}{\mathrm{min}}
\newcommand{\mut}{\mathrm{mut}}
\newcommand{\partt}{\mathrm{part}}
\newcommand{\dof}{\mathrm{dof}}
\newcommand{\NS}{\mathrm{NS}}
\newcommand{\cov}{\mathrm{cov}}
\newcommand{\gen}{\mathrm{gen}}
\newcommand{\cut}{\mathrm{cut}}
\newcommand{\parr}{\mathrm{par}}
\newcommand{\val}{\mathrm{val}}
\newcommand{\tr}{\mathrm{tr}}
\newcommand{\checkk}{\mathrm{check}}
\newcommand{\reff}{\mathrm{ref}}
\newcommand{\Mll}{M_{ll}}
\newcommand{\extra}{\mathrm{extra}}
\newcommand{\draft}[1]{}
\newcommand{\comment}[1]{{\bf \it  #1}}
\newcommand{\muf}{\mu_\text{F}}
\newcommand{\mur}{\mu_\text{R}}

\def\beq{\begin{equation}}
	\def\eeq{\end{equation}}


\def\({\left(}
\def\){\right)}
\def\[{\left[}
\def\]{\right]}
\let\originalleft\left
\let\originalright\right
\renewcommand{\left}{\mathopen{}\mathclose\bgroup\originalleft}
\renewcommand{\right}{\aftergroup\egroup\originalright}


%\let\sectionold\section
%\renewcommand\section[2][]{%
	%\sectionold{\boldmath #2}}

\let\oldsubsection\subsection
\renewcommand\subsection[2][\subsectiontoc]{%
	\def\subsectiontoc{#2}%
	\oldsubsection[#1]{\boldmath #2}%
}

\let\oldsubsubsection\subsubsection
\renewcommand\subsubsection[2][\subsubsectiontoc]{%
	\def\subsubsectiontoc{#2}%
	\oldsubsubsection[#1]{\boldmath #2}%
}


\newcommand{\tmop}[1]{\ensuremath{\operatorname{#1}}}
\newcommand{\tmtextit}[1]{{\itshape{#1}}}
\newcommand{\tmtextrm}[1]{{\rmfamily{#1}}}
\newcommand{\tmtexttt}[1]{{\ttfamily{#1}}}


\usepackage{xcolor}
\definecolor{tpurple}{RGB}{128,0,128}
\definecolor{darkgreen}{RGB}{0,180,0}
\newcommand{\SM}[1]{\textbf{\textcolor{blue}  {SM: #1}}}
\newcommand{\MB}[1]{\textbf{\textcolor{red}   {MB: #1}}}
\newcommand{\MBn}[2]{\textcolor{red}{OLD: #1 NEW: #2}}
\newcommand{\LR}[1]{{\bf\color{orange}LR: #1}}
\newcommand{\todo}[1]{{\bf\color{red}TODO: #1}}
\newcommand{\JR}[1]{{\bf\color{purple}JR: #1}}
\newcommand{\SF}[1]{{\bf\color{darkgreen}SF: #1}}
\newcommand{\RDB}[1]{{\bf\color{cyan}RDB: #1}}

\begin{document}
	
\noindent
We would like to express our gratitude to the referee for the appreciation
of our work and for providing constructive
comments on our manuscript. The valuable feedback has been used to
improve the quality and clarity of our work.
%
In response to the suggestions, we address below each of the points raised by 
the referee and describe the actions that have been taken
in the revised version of the manuscript.
%
To facilitate the review process, in the revised version of the manuscript
we have highlighted all the changes that have been made with respect to the original
submission.

\noindent

\begin{enumerate}
	\item {\it Page 3, below (2.8). It is noted that nuclear corrections are not included when 
		interpreting the neutrino structure function data in terms of proton PDFs. It would be good
		to provide some brief discussion of the size of these and the associated uncertainties.
	}
	
	In order to address the need for a comment regarding the sizes of the nuclear corrections
	in proton PDF fits, we have swapped the order of the last two sentences in that paragraph.
	The cited references, and the references therein, provide extensive discussions regarding
	the procedure to include nuclear corrections and their effects.
	
	\item {\it Table 2.2. It is not clear to me at the point where these numbers are produced exactly
		how the cross section inputs corresponding to these numbers are calculated. So I
		think a reference forward to Section 2.5, where the theory settings are described, is
		needed. Although even then, for completeness giving the PDF set that is used and
		some uncertain
		ty on the event rates here would be useful.
	}
	
	The numbers in Table 2.2 are computed by integrating the event yields shown in Eq.~2.12 in
	which the differential cross sections are computed using the central value of the PDF4LHC21
	set. For completeness, we have added a sentence above Eq. (2.16) that mention the specific
	PDF set used to compute these numbers, and where we also indicate that the calculation
        is performed at NLO in the QCD expansion.
        %
        Although
	Section 2.5 indeed provides some descriptions regarding the computation of the differential cross sections
	(which were also described in Section 2.1), it mainly focuses on the construction of the pseudodata
	needed for the fit/profiling.

        This said, we would like to emphasize that the predicted event rates in Table 2.2 depend
        only mildly on the specific settings of the DIS cross-section entering the calculation.
        %
        Furthermore, these event rates are only used to determine the expected bib-by-bin
        statistical uncertainty.
        %
        Hence our results are relatively independent on the theoretical settings
        entering the calculation of the event yields in Table 2.2.
	
	\item {\it Page 11. The discussion about consistency between the PDF set and theory settings
		used to produce the pseudodata and those entering the fit/profiling is in my view not
		correct or at least too strong. In particular, while it is perfectly reasonable to keep
		these the same there is definitely no requirement to, as is currently strongly implied in
		the discussion. In real PDF fits we often see that the fit quality for a given dataset does
		not follow textbook expectations, with $\chi^2/N \sim 1$. So some inconsistency between data
		and theory is often observed, rather than being artificial. Indeed, it is precisely because
		of this effect that tolerances (which are included in e.g. the PDF4LHC profiling) are
		included. In other words, one could perfectly reasonably generate pseudodata with a
		different PDF set, or different theory settings in order to emulate this inconsistency.
		One is free not to, but it should not be suggested that complete consistency is the only
		option here. It is a choice that is made, and not the only possible one.
	}
	
          We agree with the referee that, when generating pseudo-data, one may or may not want to
          assume the same underlying truth as in the prior of the corresponding PDF fit.
          %
          Indeed, data inconsistencies are present in real data.
          
          Our motivation to keep these settings consistent is that in this work we aim to investigate
          what is the PDF constraining power of the FPF measurements
          assuming that there are no inconsistencies,
          which is the best-case scenario.
          %
          In the presence of inconsistencies, one finds two competing effects: a shift in the central value
          of the PDF and a reduction of the PDF error whose magnitude depends on the assumed
          inconsistencies.
          %
          For example, if we generate the FPF pseudo-data far from the PDF4LHC21 prior,
          the main impact of the data will be to shift the central prediction, rather than
          to reduce the uncertainty.

          Another reason to assume this consistency  is that when carrying out Hessian profiling,
          one keeps the PDF parametrisation fixed. In the presence of inconsistencies, it may be
          that the best fit with FPF data requires a different parametrisation and hence
          a different PDF error analysis.
          %
          So Hessian profiling is most reliable in the absence of data inconsistencies.

          In the revised version of the paper, we have emphasized our two-fold region
          to assume full consistency between the prior PDF and the FPF pseudo-data: to
          quantify the PDF reach of the FPF pseudo-data in the most advantageous scenario,
          and to ensure the procedural validity of the Hessian profiling procedure.

          This said, the suggestion from the referee to generate data with an ``inconsistent'' PDF
          set is valuable and it would be interesting to carry out this exercise in the future.
          %
          To emphasize this, we have added a sentence stating that generating pseudo-data with a different
          PDF set from the prior would test the capabilities of the FPF to disentangle between
          different PDF sets among them.
          
	\item {\it Page 15, and Fig. 3.3. Perhaps some explanation of why dV and to a lesser extend uV
		benefits from charge-lepton identification could be provided?
	}
	
	We have slightly rephrased the subsequent sentences to make it clear that the improvements
	seen in $d_V$ and to a lesser extend in $u_V$ can be understood by looking at analytical 
	LO expressions of the structure functions in terms of the PDFs for $\nu$ and $\bar{\nu}$
        provided in Sect.~2.
	
	\item {\it Page 16, and Appendix A. The fact that the FASER$\nu$ (and SND@LHC) projections lead
		to a very limited improvement on the PDF uncertainties is rather hidden in a paragraph
		here, and then in the appendix. In my view, this ‘negative’ result should be given more
		prominence. It after all motivates the improvements that might come with the FPF. I
		would suggest moving this to the main body of the text and starting with this as the
		first study.
	}
	
	The main reason to put the study of the constraints provided by FASER$\nu$ in the Appendix was because
	the analysis in Section 3 is mainly dedicated to the FPF experiments. We nevertheless agree with the
	referee that such findings should be highlighted and used as a motivation for the FPF.
        %
        We have moved App. A to the first subsection of Sect 3, to highlight that the ultimate
        constraining power of LHC neutrino experiments can only be realised with the FPF.
        
        Actually, this result is also relevant concerning another points raised by the referee below:
        the fact that the FASER$\nu$ and SND@LHC data not not impact the PDFs indicates that the
        interpretation of their event rate measurements depends only loosely on the modelling
        of the deep-inelastic scattering interaction.
        %
        Hence, the FASER$\nu$ and SND@LHC data can be safely used to tune the incoming neutrino fluxes
        and learn for example about PDFs at very small-$x$, without compromising
        the validity of the interpretation.
        %
        So this ``null result'' is also relevant in this context, and motivates the use of the current
        FASER$\nu$ and SND@LHC data to validate models of charm production, forward light hadron
        production, and small-$x$ QCD.

        
	
	\item {\it Page 22. I am rather unsure about the approach for presenting results here, and in
		particular in showing numbers without including systematic errors, which are described
		as being ‘optimistic’. Having zero systematic uncertainties is surely unrealistic, rather
		than optimistic, so I feel as though a clearer justification for this needs to be given.
		Even more importantly, the labelling of the result without systematic uncertainties as
		‘FPF’ and those with as ‘FPF*’ is surely the wrong way round, given it implies that
		the case where the systematic uncertainties are not included is the default in some
		sense. So these should be swapped, and the rationale behind showing numbers without
		systematic uncertainties accounted for more clearly presented.
	}
	
	  The referee is surely correct in that zero-systematic uncertainties is not realistic.
          %
          However,
	the reason to also show results with statistical uncertainties only is twofold. First, it
	substantiates the claim that FPF measurements will be dominated by systematics, rather
        than statistical uncertainties, which is a very important and non-trivial consideration
        informing detector design. The FPF detectors have not been designed with ``precision physics'' as main
        goal, and showing that improving systematics has strong physics benefits is important
        in the context of these studies.
        %
        Note that this is not the case in FASER$\nu$ and SND@LHC, where the physics reach
        (at least concerning proton structure studies via neutrino DIS) is completely
        limited by the low statistics available.

        Second, and perhaps most importantly, we cannot claim to have a fully realistic modelling
        of systematic errors, and in particular we miss a detailed estimate of their correlation, which
        is known to be decisive in stablishing the PDF sensitivity.
        %
        Such analysis can only be carried by the experimental collaborations themselves and
        be based on a complete detector simulation, and indeed our study provides
        a strong motivation for this.
        %
        In this respect,  it is also worth noting that the estimation of the systematic uncertainties in our 
	analysis is very  much conservative, and already the insights provided by our paper
        are helping the experimentalists to improve designs to reduce systematics.
        %
        So there is a clear added value in separating the two scenarios.
        	
	Following the referee's suggestion, the results which only include the statistical errors
	are now labelled "FPF$\star$" while the one that also account for the systematics are labelled "FPF".
        %
        We have also extended the rationale for showing results in the statistics-only scenario, alongside the
        considerations described above.
	
	\item {\it Section 4 and elsewhere. Given these are HL-LHC projections, somewhere these should
		be compared with the HL-LHC PDFs of Ref [34]. This would surely be the fairer
		comparison, or in any case will give a clearer picture of where things may stand.
	}
	
	The reason why we did not provide comparisons with the HL-LHC PDFs of Ref. [34] is because
	the pseudodata used to determine these PDFs were produced with the PDF4LHC15 set which
	does not account for the recent LHC measurements. The most consistent comparisons would
	be to re-do the analysis done in ref. [34] using PDF4LHC21 and then compare the resulting PDFs 
	with our determination, something which is clearly outside the scope of this paper.

        This said, to answer the question from the referee, we have repeated the phenomenological study of Sect 4
        comparing PDF4LHC15 with PDF4LHC15+HLHC PDFs for the same processes.
        %
        Specifically, we have added to the paper an Appendix where the phenomenology study of Sect 4 is repeated
        now for the PDF4LHC15 with PDF4LHC15+HLHC PDFs.

        This comparison highlights the complementarity between the two experiments
in terms of PDF constraints, being fully orthogonal and arising from different
scattering process.
%
Furthermore, the FPF constraints benefit from the valuable property
that an eventual contamination from new physics on the PDF fits 
can be neglected, since this process is driven by $Q^2$ values outside 
the possible presence of BSM physics, at least concerning new heavy states.
%
Therefore, the ultimate sensitivity on high-precision observables being
obtained by integrating the constraints both from the HL-LHC and from the FPF
in the same global determination of PDFs.

\end{enumerate}

\noindent
We now address the comments of referee B

\begin{enumerate}
\item{\it Page 3, lines 34-36 column 2 “since in general the strange and charm … not expected to vanish”. I agree but I suggest adding a reference or a comment supporting this statement. }

  We have added the following references which explore strange and charm PDF asymmetries ~\cite{Sufian:2018cpj, Sufian:2020coz}.
  %
  We have also added a reference to the recent NNPDF study of the intrinsic charm asymmetry in the proton,
  showing that these effects are potentially sizable and could hence be relevant for the interpretation
  of the FPF measurements.

\item{\it p. 4 l. 30-35 col.2 “Also, to identify… event yields.” Why a cut on the hadronic energy should properly simulate a cut on the number of charged tracks emerging from the interaction vertex?}

The charged track multiplicity is expected to grow with the hadronic system’s invariant mass. We make this clear and cite a neutrino-Hydrogen interaction multiplicity study ~\cite{Aachen-Bonn-CERN-Munich-Oxford:1981lfk} as well as a FASER paper ~\cite{FASER:2019dxq} which also point this out. We also append this sentence:

"…, as the charged track multiplicity is expected to grow with the invariant mass of the hadronic
final state $W$~\cite{Aachen-Bonn-CERN-Munich-Oxford:1981lfk,FASER:2019dxq}."


\item{\it Table 2.1 What is the meaning of the asterisk in “FLArE (*)”? I cannot see it referred to in the caption} 

The meaning of the asterisk was to be clear that this row represents 2 proposed detectors, as opposed to the remaining rows which are each one detector. We make this clear by appending the end of the caption:

"…, which we denote for the two detectors as as FLArE(*)."

\item{\it  p.5 l. 38 col.2 “is made of thin sensitive layers” What is the material/detector the layers are made of?}

The target layers are interleaved with tungsten and emulsion. We make this clear by ammending this sentence:

"...sensitive layers of emulsion..."

\item{\it  p.5 l.49 col 1 and col 2 (“AdvSND” and “FLArE”) Can you put a reference about these experiments? Did you take the info from ref [10]?}

  We have added the original FLArE experiment proposal paper as well as the FPF whitepaper (ref[10]) to the FLArE section. For AdvSND, we have added included the reference to the FPF whitepaper, which is currently the most detailed
  description of this proposed experiment.

\item{\it p.6 l. 50-53 col.2 “Here we neglect efficiency .. simulation.” I am aware that this might be difficult to simulate at this stage but I think this effect is non-negligible due to misidentification ( a large background could enter this sample). Can you comment further on it? In addition, some guesses on misidentification capabilities can be drawn by the current FASER and SND@CERN capabilities.}

  Indeed, uncertainties due to misidentification are in general not expected to be negligible.
%
  While there are no detailed studies of charm tagging efficiencies at the FPF, as we point out in the paper, charm tagging can be done by through multiple methods, including reconstructing the topology of $D$-meson decays, and through dimuon events. With multiple techniques at hand, one should be able to determine in-situ quite well the $D$-meson tagging efficiencies, for example comparing the dimuon selection method with the charm tagging via the emulsion detector. Furthermore, the availability of different experiments based on complementary techniques will help to cross-calibrate these effects.

It is worth pointing out that our framework is flexible enough to accomodate variations in the modelling of systematic uncertainties, and in particular there is no conceptual issue preventing anyone to repeat our analysis for different treatments of efficiencies and other related experimental parameters.  


\item{\it p.9 l. 41-43 col.2 “We note that … in our estimation.” (see also the conclusions p. 27 l. 52-55 col.1) I am puzzled by this statement. It is well known that flux systematics in Faser and SND@CERN play a prominent role and set the normalization (and shape?) of eqn 2.12. How can you ignore such an important effect in your analyses? Even if the PDF constraints are marginally affected by the normalization of the neutrino flux, some of your considerations about the on-axis (faser and faser2) versus the off-axis experiments (SND@CERN and its upgrades) may be affected by flux uncertainties. Can you comment on it in the paper?}

  Indeed, the large neutrino flux uncertainties are well established, and the shape and normalization of the flux can vary widely between different models, although the neutrino muon component which is mostly relevant for neutrino DIS experiments is the one which is known the best.

  It is important to note that neutrino measurements actually constrain the product of flux and cross-section - each of these components brings an uncertainty with it, and a full analysis with real data would constrain them simultaneously. We wish to motivate this joint analysis by calculating the  impact that FPF data can bring to a PDF fit. In this sense, by taking the flux to be known we can understand the full reach of FPF data on cross-section measurements, analogous to what was done in 2309.10417. Moreover, projections of FPF data on flux measurements has shown that HL-LHC data can bring flux uncertainties to a sub-percent level. 


  We would like to raise two more considerations which justify, in this work, to assume that the incoming neutrino
  fluxes are known:
  \begin{itemize}
  \item As shown in appendix A in the original version of the paper (now moved to Sect 3), FASER$\nu$ and SND@LHC data from run III do not have sensitivity on the modelling of the neutrino-nucleus interaction cross-section.
    %
    This means that these measurements can be safely used, before the start of the FPF, to cross-calibrate
    incoming neutrino fluxes and markedly improve their predictions for subsequent analysis.
    %
    Just due to this, by the time the FPF starts data taking the neutrino fluxes will be known
    more precisely than now.

  \item The flux and the neutrino DIS cross-section have very different kinematic dependences.
    %
    The incoming nuetrino flux is mostly fixed from the $(E_\nu, y_\nu)$ dependence of the event rates,
    while the PDFs in the DIS cross-section are fixed by the $(x,Q^2)$ dependence.
    %
    By measuring event rates differentially in the three kinematic variables $(E_\nu, x,Q^2)$, and also
    accessing the $y_\nu$ dependence of the incoming neutrinos, one can efficiency disentangle the
    fluxes from the DIS cross-section.
    %
    Of course, one needs to explicitely demonstrate this via a dedicated analysis, but there
    is no conceptual reason preventing us from this goal.

    
   \end{itemize} 

We thank the referee for this important point and add the following paragraph after we introduce the flux model in the middle of section 2.3. 

  "%
As pointed out in Ref.~\cite{Kling:2021gos} there are notable neutrino flux uncertainties, as various event generators do not agree on the forward parent hadron spectra. 
%
If the spread of various generators' predictions was taken as a means of flux uncertainty, corresponding to a $\lesssim 50\%$ uncertainty on the interacting muon neutrino spectrum, this would be a significant systematic if left unresolved. However, it is noteworthy that many existing predictions are yet to be tuned for the purposes of experiments such as those planned for the FPF. Nevertheless, there are projections of FPF measurements which would reduce this uncertainty to the sub-percent level already in the context of the contemporary predictions, based on parametrizing their expected correlations~\cite{Kling:2023tgr}, as well as efforts to describe the uncertainty in a data-driven way while improving the modelling of forward hadronization~\cite{Fieg:2023kld}.
%
However, it is important to note that forward neutrino experiments actually constrain the product of flux and cross-section, and one must be assumed to measure the other. In a full analysis,  the flux and cross-section would be constrained simultaneously in a joint measurement, utilizing their different kinematic dependences on $ x,Q^2,E_{\nu}$ and neutrino rapidity $\eta_{\nu}$.
%
In our study, we aim to understand the full impact of FPF data on the PDF fit, thus motivating this future joint measurement.
%
To this aim, we take the neutrino flux to be known and focus on the irreducible systematics associated with event reconstruction. 
%
With this assumption, we will show that Run 3 measruements will not be sufficient to impact PDF fits. Instead, Run 3 measurements could be used to calibrate incoming neutrino fluxes, effectively reducing the large uncertainties by the time FPF data is collected in the future. The expected reduction of FPF neutrino flux uncertainties further justifies our choice to take the neutrino flux as known.
%"

\item{\it p.11 l. 33-37 col.1 Are you sure that an assumption of $f_{corr}$ = 0.5 is realistic? The estimates made in ref.[34] are for LHC experiments, not for fixed-target experiments like those of the FPF.}

  In the absence of full detector simulation and detailed correlation model, it is not possible to answer
  this point conclusively.
  %
  What we can say is that we have repeated our analysis for $f_{corr}=1$ and find that our main
  qualitative conclusions remain unchanged.

\item{\it  Fig. 2.4 In the bottom-left plot, I see a sharp increase in the fractional error at about x= 4 e-2. Why?}

  The points near $x$=4e-2 with large uncertainties have small $E_h$, and are approaching the $E_h>100~ {\rm GeV}$ threshold that we set in Table 2.1. When we fluctuate the data according to our definition of uncertainty (Eq. 2.19 with $E_h$ instead of $E_{\ell}$), these points drop below the acceptance and thus contribute to a fluctuation in this bin.
  %
  We have added to the revised version of the paper a short discussion to explain this observation.

\item{\it p.19 l.50-53 col 1 and in general along the paper: I cannot see a discussion of the impact of neutral-current events at the FPF. Can you further comment on it?}

  There are expected to be roughly as many neutral current neutrino scattering events as charged current events. However, due to the lack of information on both the incoming and outgoing neutrino, the full event cannot be reconstructed, only the total hadronic energy can be measured. One could try to use the total NC event rate as a means of constraining the integrated PDF, however we expect that this would negligibly improve the impact on the PDF constraints as compared to fully reconstructed CC events. Indeed, if anything, NC scattering brings in less sensitivity to PDFs than CC events, as can be seen from the LO decomposition of the corresponding structure functions.

  We have added to Sect 2 a short discussion on why it is very likely that NC events do not provide useful PDF information: first of all, event kinematics cannot be reconstructed on an event by event basis as in the CC case, second, that systematics will be large since are dominated by the hadronic final state, and third, that NC scattering in general provides less complete information on the PDF flavour decomposition as compared to CC scattering. 

\end{enumerate}
\bibliographystyle{utphys}
\bibliography{main.bib}
\end{document}
