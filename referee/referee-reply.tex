\documentclass[11pt,a4paper]{article}
\usepackage{placeins}
\usepackage{graphicx}
\usepackage{xcolor}
\usepackage{float}
\usepackage{afterpage}
\usepackage{amssymb,amsmath}
\usepackage{multirow,booktabs,multirow}
\usepackage{cite}
\usepackage[colorlinks=true, linkcolor=black!50!blue, urlcolor=blue, citecolor=blue, anchorcolor=blue]{hyperref}
\usepackage[font=small,labelfont=bf,margin=0mm,labelsep=period,tableposition=top]{caption}
\usepackage[a4paper,top=3cm,bottom=2.5cm,left=2.5cm,right=2.5cm,bindingoffset=0mm]{geometry}
\setlength{\unitlength}{1mm}

\usepackage{tabularx}
\newcolumntype{C}[1]{>{\centering\arraybackslash}p{#1}}

%%%%%%%%%%%%%%%%%%%%%%%%%%%%%%%%%%%%%%%%%%%%%%%%%%%%%%%%%%%%%

\def\smallfrac#1#2{\hbox{$\frac{#1}{#2}$}}
\newcommand{\be}{\begin{equation}}
	\newcommand{\ee}{\end{equation}}
\newcommand{\bea}{\begin{eqnarray}}
	\newcommand{\eea}{\end{eqnarray}}
\newcommand{\bi}{\begin{itemize}}
	\newcommand{\ei}{\end{itemize}}
\newcommand{\ben}{\begin{enumerate}}
	\newcommand{\een}{\end{enumerate}}
\newcommand{\la}{\left\langle}
\newcommand{\ra}{\right\rangle}
\newcommand{\lc}{\left[}
\newcommand{\rc}{\right]}
\newcommand{\lp}{\left(}
\newcommand{\rp}{\right)}
\newcommand{\as}{\alpha_s}
\newcommand{\aq}{\alpha_s\left( Q^2 \right)}
\newcommand{\amz}{\alpha_s\left( M_Z^2 \right)}
\newcommand{\aqq}{\alpha_s \left( Q^2_0 \right)}
\newcommand{\aqz}{\alpha_s \left( Q^2_0 \right)}
\newcommand{\Ord}{\mathcal{O}}
\newcommand{\MSbar}{\overline{\text{MS}}}
\def\toinf#1{\mathrel{\mathop{\sim}\limits_{\scriptscriptstyle
			{#1\rightarrow\infty }}}}
\def\tozero#1{\mathrel{\mathop{\sim}\limits_{\scriptscriptstyle
			{#1\rightarrow0 }}}}
\def\toone#1{\mathrel{\mathop{\sim}\limits_{\scriptscriptstyle
			{#1\rightarrow1 }}}}
\def\frac#1#2{{{#1}\over {#2}}}
\def\gsim{\gtrsim}
\def\lsim{\lesssim}
\newcommand{\mrexp}{\mathrm{exp}}
\newcommand{\dat}{\mathrm{dat}}
\newcommand{\one}{\mathrm{(1)}}
\newcommand{\two}{\mathrm{(2)}}
\newcommand{\art}{\mathrm{art}} 
\newcommand{\rep}{\mathrm{rep}}
\newcommand{\net}{\mathrm{net}}
\newcommand{\stopp}{\mathrm{stop}}
\newcommand{\sys}{\mathrm{sys}}
\newcommand{\stat}{\mathrm{stat}}
\newcommand{\diag}{\mathrm{diag}}
\newcommand{\pdf}{\mathrm{pdf}}
\newcommand{\tot}{\mathrm{tot}}
\newcommand{\minn}{\mathrm{min}}
\newcommand{\mut}{\mathrm{mut}}
\newcommand{\partt}{\mathrm{part}}
\newcommand{\dof}{\mathrm{dof}}
\newcommand{\NS}{\mathrm{NS}}
\newcommand{\cov}{\mathrm{cov}}
\newcommand{\gen}{\mathrm{gen}}
\newcommand{\cut}{\mathrm{cut}}
\newcommand{\parr}{\mathrm{par}}
\newcommand{\val}{\mathrm{val}}
\newcommand{\tr}{\mathrm{tr}}
\newcommand{\checkk}{\mathrm{check}}
\newcommand{\reff}{\mathrm{ref}}
\newcommand{\Mll}{M_{ll}}
\newcommand{\extra}{\mathrm{extra}}
\newcommand{\draft}[1]{}
\newcommand{\comment}[1]{{\bf \it  #1}}
\newcommand{\muf}{\mu_\text{F}}
\newcommand{\mur}{\mu_\text{R}}

\def\beq{\begin{equation}}
	\def\eeq{\end{equation}}


\def\({\left(}
\def\){\right)}
\def\[{\left[}
\def\]{\right]}
\let\originalleft\left
\let\originalright\right
\renewcommand{\left}{\mathopen{}\mathclose\bgroup\originalleft}
\renewcommand{\right}{\aftergroup\egroup\originalright}


%\let\sectionold\section
%\renewcommand\section[2][]{%
	%\sectionold{\boldmath #2}}

\let\oldsubsection\subsection
\renewcommand\subsection[2][\subsectiontoc]{%
	\def\subsectiontoc{#2}%
	\oldsubsection[#1]{\boldmath #2}%
}

\let\oldsubsubsection\subsubsection
\renewcommand\subsubsection[2][\subsubsectiontoc]{%
	\def\subsubsectiontoc{#2}%
	\oldsubsubsection[#1]{\boldmath #2}%
}


\newcommand{\tmop}[1]{\ensuremath{\operatorname{#1}}}
\newcommand{\tmtextit}[1]{{\itshape{#1}}}
\newcommand{\tmtextrm}[1]{{\rmfamily{#1}}}
\newcommand{\tmtexttt}[1]{{\ttfamily{#1}}}


\usepackage{xcolor}
\definecolor{tpurple}{RGB}{128,0,128}
\definecolor{darkgreen}{RGB}{0,180,0}
\newcommand{\SM}[1]{\textbf{\textcolor{blue}  {SM: #1}}}
\newcommand{\MB}[1]{\textbf{\textcolor{red}   {MB: #1}}}
\newcommand{\MBn}[2]{\textcolor{red}{OLD: #1 NEW: #2}}
\newcommand{\LR}[1]{{\bf\color{orange}LR: #1}}
\newcommand{\todo}[1]{{\bf\color{red}TODO: #1}}
\newcommand{\JR}[1]{{\bf\color{purple}JR: #1}}
\newcommand{\SF}[1]{{\bf\color{darkgreen}SF: #1}}
\newcommand{\RDB}[1]{{\bf\color{cyan}RDB: #1}}

\begin{document}
	
\noindent
We would like to express our gratitude to the referee for the appreciation
of our work and for providing constructive
comments on our manuscript. The valuable feedback has been used to
improve the quality and clarity of our work.
%
In response to the suggestions, we address below each of the points raised by 
the referee and describe the actions that have been taken
in the revised version of the manuscript.

\noindent

\begin{enumerate}
	\item {\it Page 3, below (2.8). It is noted that nuclear corrections are not included when 
		interpreting the neutrino structure function data in terms of proton PDFs. It would be good
		to provide some brief discussion of the size of these and the associated uncertainties.
	}
	
	$\cdots$
	
	\item {\it able 2.2. It is not clear to me at the point where these numbers are produced exactly
		how the cross section inputs corresponding to these numbers are calculated. So I
		think a reference forward to Section 2.5, where the theory settings are described, is
		needed. Although even then, for completeness giving the PDF set that is used and
		some uncertain
		ty on the event rates here would be useful.
	}
	
	$\cdots$
	
	\item {\it Page 11. The discussion about consistency between the PDF set and theory settings
		used to produce the pseudodata and those entering the fit/profiling is in my view not
		correct or at least too strong. In particular, while it is perfectly reasonable to keep
		these the same there is definitely no requirement to, as is currently strongly implied in
		the discussion. In real PDF fits we often see that the fit quality for a given dataset does
		not follow textbook expectations, with $\chi^2/N \sim 1$. So some inconsistency between data
		and theory is often observed, rather than being artificial. Indeed, it is precisely because
		of this effect that tolerances (which are included in e.g. the PDF4LHC profiling) are
		included. In other words, one could perfectly reasonably generate pseudodata with a
		different PDF set, or different theory settings in order to emulate this inconsistency.
		One is free not to, but it should not be suggested that complete consistency is the only
		option here. It is a choice that is made, and not the only possible one.
	}
	
	$\cdots$
	
	\item {\it Page 15, and Fig. 3.3. Perhaps some explanation of why dV and to a lesser extend uV
		benefits from charge-lepton identification could be provided?
	}
	
	$\cdots$
	
	\item {\it Page 16, and Appendix A. The fact that the FASER$\nu$ (and SNDLHC) projections lead
		to a very limited improvement on the PDF uncertainties is rather hidden in a paragraph
		here, and then in the appendix. In my view, this ‘negative’ result should be given more
		prominence. It after all motivates the improvements that might come with the FPF. I
		would suggest moving this to the main body of the text and starting with this as the
		first study.
	}
	
	$\cdots$
	
	\item {\it Page 22. I am rather unsure about the approach for presenting results here, and in
		particular in showing numbers without including systematic errors, which are described
		as being ‘optimistic’. Having zero systematic uncertainties is surely unrealistic, rather
		than optimistic, so I feel as though a clearer justification for this needs to be given.
		Even more importantly, the labelling of the result without systematic uncertainties as
		‘FPF’ and those with as ‘FPF*’ is surely the wrong way round, given it implies that
		the case where the systematic uncertainties are not included is the default in some
		sense. So these should be swapped, and the rationale behind showing numbers without
		systematic uncertainties accounted for more clearly presented.
	}
	
	The referee is surely correct in that zero-systematic uncertainties is unrealistic. However,
	the reason to also show results with statistical uncertainties only is twofold. First, it
	substantiates the claim that FPF measurements will be statistically dominated. Second, in the
	ideal scenario (most optimistic) in which all the sources of systematic uncertainties are 
	under control, the statistic-only case would be the limit in terms of effects. It is also 
	worth noting that the estimation of the systematic uncertainties in our analysis is very 
	much conservative.
	
	According to the referee's suggestion, the results which only includes the statistical errors
	are labelled "FPF$\star$" while the one that also account for the systematics are labelled "FPF".
	
	\item {\it Section 4 and elsewhere. Given these are HL-LHC projections, somewhere these should
		be compared with the HL-LHC PDFs of Ref [34]. This would surely be the fairer
		comparison, or in any case will give a clearer picture of where things may stand.
	}
	
	$\cdots$

\end{enumerate}
	
	
\end{document}
