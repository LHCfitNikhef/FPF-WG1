\section{Introduction and motivation}
\label{sec:introduction}

Proton-proton collisions at energy-frontier hadron colliders
such as the LHC produce an intense collimated flux of neutrinos~\cite{Kling:2021gos}.
%
These neutrinos are characterised by the largest energies ever achieved in laboratory experiments,
reaching up to several TeV.
%
Due to the lack of dedicated instrumentation in the LHC far-forward region,
until recently these neutrinos escaped detection.
%
The reported observation of LHC neutrinos~\cite{FASER:2023zcr,SNDLHC:2023pun} by the
far-forward  FASER$\nu$~\cite{FASER:2019dxq} and SND@LHC~\cite{SHiP:2020sos,SNDLHC:2022ihg} experiments
demonstrates that this hitherto discarded beam can now be deployed for physics studies.
%
Beyond the ongoing Run III, a dedicated suite of upgraded far-forward
neutrino experiments hosted by a
Forward Physics Facility (FPF)~\cite{Anchordoqui:2021ghd,Feng:2022inv} has been
proposed to operate concurrently with the High-Luminosity LHC~\cite{Azzi:2019yne,Cepeda:2019klc}.
%
LHC neutrinos enable
unprecedented scientific opportunities
for particle and astroparticle physics both within the Standard Model and beyond it,
as summarised in~\cite{Anchordoqui:2021ghd,Feng:2022inv}
and references therein.

As well known~\cite{Conrad:1997ne,Mangano:2001mj},
measurements of neutrino structure functions~\cite{Candido:2023utz} in deep-inelastic scattering
(DIS) provide sensitive information on the parton distribution functions (PDFs)
of nucleons and nuclei~\cite{Ethier:2020way,Gao:2017yyd,Kovarik:2019xvh}, in particular
they probe the quark/antiquark  flavour separation including
strangeness~\cite{NuTeV:2007uwm,CCFR:1994ikl,Faura:2020oom,NOMAD:2013hbk}.
%
These constraints, arising from charged-current DIS, are fully complementary to those
provided by neutral-current charged-lepton DIS given that they reveal independent
flavour combinations.
%
Several  experiments have measured neutrino-nucleus
DIS structure functions over a wide range of energies, and neutrino data
from the CHORUS~\cite{CHORUS:2005cpn}, NuTeV~\cite{NuTeV:2001dfo},
CCFR~\cite{Yang:2000ju}, and CDHS~\cite{Berge:1989hr}
and other experiments is routinely  included in global
determinations of proton~\cite{NNPDF:2021njg,Hou:2019efy,Bailey:2020ooq} and nuclear
PDFs~\cite{Eskola:2021nhw,AbdulKhalek:2022fyi,Muzakka:2022wey}.

As compared to these previous experiments,
the unique benefit of LHC neutrinos is their energy spectrum, reaching
energies up to a factor 10 higher (several TeV).
%
This feature, together with the large statistics (up to $10^6$ muon neutrinos detected at the FPF~\cite{Kling:2021gos}),  strongly suggest that the LHC  neutrino beam should be
deployed as a precise probe of hadron structure, complementing
and extending available results by an order of magnitude both at small-$x$
and large-$Q^2$~\cite{Feng:2022inv}.
%
However, we currently lack  dedicated projections for the kinematic coverage
and accuracy expected for LHC neutrino structure functions,
including bin-by-bin statistical and systematic uncertainties and their
correlations, for the various operating and proposed detectors.
%
The lack of these projections
has so far prevented quantitative studies assessing the impact
of LHC neutrinos in global fits of proton and nuclear PDFs along the lines
of those performed for the HL-LHC~\cite{AbdulKhalek:2018rok}, the Electron-Ion Collider (EIC)~\cite{AbdulKhalek:2021gbh,Khalek:2021ulf}, and the
Large Hadron electron Collider (LHeC)~\cite{AbdulKhalek:2019mps,LHeC:2020van}. 

The goal of this work is to bridge this gap and quantify
the expected impact of  DIS structure functions from the LHC
neutrino beam on the proton and nuclear PDFs.
%
To this end, we carry out projections for  FASER$\nu$ and SND@LHC based on the full Run III luminosity
as well as for  three of the proposed experiments to be hosted
at the FPF~\cite{Anchordoqui:2021ghd,Feng:2022inv}: FLArE, AdvSND, and FASER$\nu$2.
%
For each experiment, we determine the expected event yields in bins in $x$, $Q^2$,
and neutrino energy $E_\nu$,
generate pseudo-data for  inclusive and charm 
structure functions, 
and estimate the main systematic uncertainties and their correlation pattern.
%
Subsequently, we study their impact on the proton PDFs by means of the Hessian profiling~\cite{Paukkunen:2014zia}
of the PDF4LHC21 combination~\cite{PDF4LHCWorkingGroup:2022cjn}
within the  {\sc\small xFitter} framework~\cite{Alekhin:2014irh}, cross-checked
with their direct inclusion in the open-source NNPDF fitting framework~\cite{NNPDF:2021uiq}.
%
We also assess their impact on the PDFs of tungsten (target of FASER$\nu$ and SND@LHC)
using the EPPS21 global nuclear PDF fit~\cite{Eskola:2021nhw}
profiled with {\sc\small xFitter}.

Our analysis reveals that  neutrino structure function measurements at the LHC, specially
for the high-statistics  experiments of the FPF, provides  stringent constraints
on the quark and antiquark (nuclear) PDFs, specially on the up and down
antiquarks and on strangeness, as compared to state-of-the-art (n)PDF determinations.
%
The best-case scenario when only statistical errors are considered is not
significantly distorted upon the inclusion of realistic systematic errors.
%
We make the resulting (n)PDF sets publicly available, as well as the DIS pseudo-data projections
for the various scenarios for the systematic uncertainties considered.
%
Our work demonstrates that detection of LHC far-forward neutrinos  effectively
extends the LHC with a charged-current counterpart of the EIC,
with similar kinematic reach and complementary charged-current reactions,
therefore realising (upon Lorentz-boosting) the analog of a ``Neutrino-Ion Collider'' at CERN
without the need of additional accelerator infrastructure.


The outline of this paper is as follows.
%
First, in Sect.~\ref{sec:dis_pseudodata} we describe the procedure
adopted to generate projections for DIS structure functions associated to
LHC neutrino scattering, including the modelling of experimental systematic
uncertainties.
%
Then in Sect.~\ref{sec:settings} we discuss the theoretical settings used
for the calculation of DIS structure functions and the methodology adopted
to account for the constraints of LHC neutrino DIS data on proton and nuclear PDFs.
%
The projected impact of these structure functions on proton and nuclear
PDFs is quantified in Sects.~\ref{sec:protonPDFs} and~\ref{sec:nuclearPDFs} respectively.
%
Finally we summarise in Sect.~\ref{sec:summary} and consider some possible
directions for follow-up research.
