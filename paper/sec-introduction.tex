\section{Introduction and motivation}
\label{sec:introduction}

Proton-proton collisions at energy-frontier hadron colliders
such as the LHC produce an intense collimated flux of neutrinos.
%
These neutrinos are characterised by the largest energies ever achieved in laboratory experiments and
reaching up to several TeV~\cite{Kling:2021gos}.
%
Due to the lack of dedicated instrumentation in the LHC far-forward region,
until very recently these neutrinos escaped detection.
%
The reported observation of LHC neutrinos~\cite{FASER:2023zcr,SNDLHC:2023pun} by the
far-forward  FASER$\nu$~\cite{FASER:2019dxq} and SND@LHC~\cite{SHiP:2020sos,SNDLHC:2022ihg} experiments
demonstrates that this hitherto discarded beam can now be deployed for physics studies.
%
Beyond the ongoing Run III, a dedicated suite of upgraded far-forward
neutrino experiments hosted by a
Forward Physics Facility (FPF)~\cite{Anchordoqui:2021ghd,Feng:2022inv} has been
proposed to operate concurrently with the High-Luminosity LHC~\cite{Azzi:2019yne,Cepeda:2019klc}.
%
Current and future 
LHC neutrinos experiments enable
unprecedented scientific opportunities
for particle and astroparticle physics both within the Standard Model and beyond it,
as summarised in~\cite{Anchordoqui:2021ghd,Feng:2022inv}
and references therein.

As well known~\cite{Conrad:1997ne,Mangano:2001mj},
measurements of neutrino structure functions~\cite{Candido:2023utz} in deep-inelastic scattering
(DIS) provide sensitive information on the parton distribution functions (PDFs)
of nucleons and nuclei~\cite{Ethier:2020way,Gao:2017yyd,Kovarik:2019xvh}, and in particular
they probe the quark/antiquark  flavour separation including
strangeness~\cite{NuTeV:2007uwm,CCFR:1994ikl,Faura:2020oom,NOMAD:2013hbk}.
%
These constraints, arising from charged-current DIS, are fully complementary to those
provided by neutral-current charged-lepton DIS, given that they reveal independent
quark flavour combinations.
%
Several  experiments have measured neutrino-nucleus
DIS structure functions over a wide range of energies, and neutrino data
from the CHORUS~\cite{CHORUS:2005cpn}, NuTeV~\cite{NuTeV:2001dfo},
CCFR~\cite{Yang:2000ju}, and CDHS~\cite{Berge:1989hr}
and other experiments is routinely  included in global
determinations of proton~\cite{NNPDF:2021njg,Hou:2019efy,Bailey:2020ooq} and nuclear
PDFs~\cite{Eskola:2021nhw,AbdulKhalek:2022fyi,Muzakka:2022wey}.

As compared to  previous neutrino DIS experiments,
the unique benefit of LHC neutrinos is their energy spectrum, reaching
energies up to an order of magnitude higher.
%
This feature, together with the large statistics (with up to one million muon neutrinos to be
detected at the FPF~\cite{Kling:2021gos}),  indicates that the LHC  neutrino beam should be
deployed as a precise probe of hadron structure in order to complement
and extend available results by a factor 10 both at small-$x$
and large-$Q^2$~\cite{Feng:2022inv}.
%
However, we currently lack  dedicated projections for the kinematic coverage
and experimental accuracy expected for LHC neutrino structure functions,
including bin-by-bin statistical and systematic uncertainties and their
correlations, for the various operating and proposed detectors.
%
The lack of these projections
has so far prevented quantitative studies assessing the impact
of LHC neutrinos in global fits of proton and nuclear PDFs along the lines
of those performed for the HL-LHC~\cite{AbdulKhalek:2018rok}, the Electron-Ion Collider (EIC)~\cite{AbdulKhalek:2021gbh,Khalek:2021ulf}, and the
Large Hadron electron Collider (LHeC)~\cite{AbdulKhalek:2019mps,LHeC:2020van}. 

The goal of this work is to bridge this gap and quantify
the expected impact of  DIS structure functions from the LHC
neutrino experiments on the proton and nuclear PDFs.
%
To this end, we carry out projections for  FASER$\nu$ and SND@LHC based on the full Run III luminosity
as well as for  three of the proposed experiments to be hosted
at the FPF~\cite{Anchordoqui:2021ghd,Feng:2022inv}: FLArE, AdvSND, and FASER$\nu$2.
%
For each experiment, we determine the expected event yields in bins in $x$, $Q^2$,
and neutrino energy $E_\nu$,
generate pseudo-data for  inclusive and charm 
structure functions, 
and estimate the main systematic uncertainties.
%
Subsequently, we study their impact on the proton PDFs by means of both the Hessian profiling~\cite{Paukkunen:2014zia,  Schmidt:2018hvu, AbdulKhalek:2018rok, HERAFitterdevelopersTeam:2015cre}
of the PDF4LHC21 combination~\cite{PDF4LHCWorkingGroup:2022cjn}
within the  {\sc\small xFitter} framework~\cite{Alekhin:2014irh, Bertone:2017tig, xFitter:2022zjb, xFitter:web}
as well as the direct inclusion in the open-source NNPDF4.0 fitting framework~\cite{NNPDF:2021uiq}.
%
We also assess their impact on the PDFs of tungsten, the target material of FASER$\nu$ and SND@LHC,
using the EPPS21 global nuclear PDF fit~\cite{Eskola:2021nhw}
profiled with {\sc\small xFitter}.

Our analysis reveals that  LHC neutrino structure function measurements at the FPF
would  provide  stringent constraints
on the quark and antiquark PDFs, specially on the up and down
valence quarks and on strangeness, as compared to state-of-the-art determinations.
%
We find that accounting for systematic uncertainties does not significantly
degrade the sensitivity achieved by the statistics-only baseline.
%
We also quantify the impact of charm-tagged structure functions, study the relevance
of final-state lepton charge identification, and compare the relative sensitivity
provided by each of the FPF experiments.

Together with this paper, we release the generated  DIS pseudo-data projections for the various scenarios
considered, as well as the resulting PDF sets, for which one can foresee
many other applications beyond hadronic structure studies.
%
Our work demonstrates that the availability of far-forward neutrino detection
at the LHC effectively
extends CERN with a charged-current counterpart of the EIC,
with a similar kinematic reach and complementing
the information provided by its neutral-current scattering data.
%
Therefore, LHC neutrino experiments realise, upon Lorentz-boosting, the analog of
a ``Neutrino-Ion Collider'' at CERN
without the need of additional accelerator infrastructure or energy consumption.

The outline of this paper is as follows.
%
Sect.~\ref{sec:dis_pseudodata} discusses the procedure
adopted to generate projections for neutrino DIS structure functions at the LHC,
as well as the methodology used
to include them in proton and nuclear PDF fits.
%
The projected impact of these structure functions on proton and nuclear
PDFs is quantified in Sect.~\ref{sec:protonPDFs}.
%
We summarise in Sect.~\ref{sec:summary} and also consider some possible
directions for follow-up research.
%
Additional impact results are collected in two appendices:
App.~\ref{app:fasernu_runIII_impact} studies the possible PDF impact
of FASER$\nu$ Run III measurements, and App.~\ref{app:nPDF_impact_appendix}
quantifies the stability of the nuclear PDF projections with respect
to variations of its inputs.
