\section{Introduction and motivation}
\label{sec:introduction}

Proton-proton collisions at energy-frontier hadron colliders
such as the LHC produce an intense collimated flux of neutrinos~\cite{Kling:2021gos}.
%
These neutrinos are characterised by the largest energies ever achieved in laboratory experiments
and reaching up to  several TeV, only surpassed by astrophysical neutrinos.
%
Due to the lack of dedicated instrumentation in the LHC far-forward region,
until recently these neutrinos escaped detection and hence could not be used
for scientific enquiry.
%
This situation changed drastically in  March 2023
with the reports of the first observation of LHC neutrinos~\cite{FASER:2023zcr} by the
far-forward  FASER$\nu$~\cite{FASER:2019dxq} and SND@LHC~\cite{SHiP:2020sos} experiments.
%
The discovery of LHC neutrinos enables
unprecedented scientific opportunities
for particle and astroparticle physics both within the Standard Model and beyond it,
see~\cite{Anchordoqui:2021ghd,Feng:2022inv}
and references therein for a comprehensive summary.
%
Beyond the ongoing Run III, a dedicated suite of upgraded far-forward
neutrino experiments hosted by a dedicated
Forward Physics Facility (FPF)~\cite{Anchordoqui:2021ghd,Feng:2022inv} has been
proposed to operate concurrently with the HL-LHC.

Among its many possible applications, the high-energy neutrino beam produced by the LHC
can be deployed as a sensitive probe of hadron structure.
%
As well known~\cite{Conrad:1997ne,Mangano:2001mj},
measurements of neutrino structure functions in deep-inelastic scattering
(DIS)~\cite{Candido:2023utz} provide valuable information on the parton distribution functions (PDFs)
of nucleons and nuclei~\cite{Ethier:2020way,Gao:2017yyd,Kovarik:2019xvh}, in particular
concerning the quark and antiquark PDF flavour separation.
%
Furthermore, these constraints from charged-current neutrino DIS are complementary to those
provided by neutral-current charged-lepton DIS, being sensitive to independent
flavour combinations.
%
With this motivation, several  experiments have in the past measured neutrino-nucleus
DIS structure functions over a wide range of energies, and neutrino data
from e.g. the CHORUS~\cite{CHORUS:2005cpn}, NuTeV~\cite{NuTeV:2001dfo},
CCFR~\cite{Yang:2000ju}, and CDHS~\cite{Berge:1989hr} experiments enter 
some of the most recent global determinations of proton~\cite{Hou:2019efy,Bailey:2020ooq,NNPDF:2021njg} and
nuclear PDFs~\cite{AbdulKhalek:2022fyi,Eskola:2021nhw,Muzakka:2022wey}.

As compared to previous neutrino DIS measurements,  LHC data benefits
from higher neutrino energies  increasing the kinematic coverage
in the $(x,Q^2)$ plane by an order of magnitude both in the small-$x$
and the large-$Q^2$ directions~\cite{Feng:2022inv}.
%
Furthermore, the large number of expected neutrino scattering events enable
structure function measurements with excellent precision,
e.g. the  $\mathcal{O}(10^6)$ muon neutrinos expected at the FPF detectors~\cite{Kling:2021gos}
enable sub-percent statistical uncertainties for most of the bins in the $(x,Q^2)$ plane.
%
However,  so far discussions about the impact of far-forward LHC neutrinos on hadronic structure
have been restricted to the level  of qualitative estimates.
%
The main reason for this limitation is the lack
of dedicated projections for the LHC neutrino structure function data, including the expected
statistical and systematic uncertainties for the various operating and proposed detectors.

The main goal of this work is to bridge this gap and quantify
the expected impact of LHC neutrino DIS data on the PDFs of  nucleons and heavy nuclei
for the various relevant experiments and for different assumptions about their performance
and the integrated luminosities.
%
In addition to projections for FASER$\nu$ and SND@LHC based on the full Run III luminosity,
we consider also three of the proposed FPF experiments, FLArE, AdvSND, and FASER$\nu$2.
%
For each detector, we generate dedicated  pseudo-data for both inclusive and charm (dimuon)
structure functions and estimate the event yields as a function of the binning
in $x$ and $Q^2$.
%
Emphasis is devoted to assessing the expected systematic uncertainties
from the intended acceptance and performance of the LHC neutrino experiments.

Following this DIS pseudo-data generation, we study their impact first
on the proton PDFs and subsequently on the nuclear PDFs.
%
In the former case, this is achieved both by means of the Hessian profiling~\cite{Paukkunen:2014zia}
of the PDF4LHC21 combination~\cite{PDF4LHCWorkingGroup:2022cjn}
within the  {\sc\small xFitter} framework~\cite{Alekhin:2014irh} and by means
of their direct inclusion in the open-source NNPDF fitting framework~\cite{NNPDF:2021uiq}.
%
In the latter case, the EPPS21 global nuclear PDF determination~\cite{Eskola:2021nhw}
is profiled again using {\sc\small xFitter}.
%
Theoretical predictions for neutrino DIS structure functions are evaluated with the {\sc\small EKO}~\cite{Candido:2022tld}
and {\sc\small YADISM}~\cite{yadism,Candido:2023utz} programs interfaced to {\sc\small PineAPPL}~\cite{Carrazza:2020gss}, with the latter providing the fast interpolation grids to be used in the fit.
%
To this end, a new interface between  {\sc\small PineAPPL} and {\sc\small xFitter} has been developed.

Qualitatively consistent results are obtained from the analyses using both the {\sc\small xFitter}
and NNPDF frameworks and reveal that  LHC neutrino scattering, specially
for the high-statistics  experiments of the FPF, provides  stringent constraints
disentangling the quark and antiquark (nuclear) PDFs of different flavours.
%
We find that the inclusion of systematic uncertainties does not significantly modify the
impact of these structure function measurements on the PDFs as compared
to an analysis that considers only statistical errors, unless extreme assumptions on the bin-by-bin
correlation model are adopted.
%
We illustrate the impact that the expected improvements on the (nuclear) PDFs provided 
on LHC neutrinos may have on precision measurements at the HL-LHC.
%
Our analysis demonstrates that detection and analysis of LHC far-forward neutrinos  effectively
extends the LHC with a charged-current counterpart of the
US-based Electron-Ion Collider (EIC)~\cite{AbdulKhalek:2021gbh}
with comparable kinematic reach, therefore realising a ``Neutrino-Ion Collider'' at CERN
without the need of additional accelerator infrastructure.


The outline of this paper is as follows.
%
First of all, in Sect.~\ref{sec:dis_pseudodata} we describe the procedure
adopted to generate projections for DIS structure functions associated to
LHC neutrino scattering, including the modelling of experimental systematic
uncertainties.
%
Then in Sect.~\ref{sec:settings} we discuss the theoretical settings used
for the calculation of DIS structure functions and the methodology adopted
to account for the constraints of LHC neutrino DIS data on proton and nuclear PDFs.
%
The projected impact of these structure functions on proton and nuclear
PDFs is quantified in Sects.~\ref{sec:protonPDFs} and~\ref{sec:nuclearPDFs} respectively.
%
Finally we summarise in Sect.~\ref{sec:summary} and consider some possible
directions for follow-up research.
