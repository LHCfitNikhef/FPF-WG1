\section{Introduction and motivation}
\label{sec:introduction}

Proton-proton collisions at the LHC produce a high-intensity collimated flux of neutrinos.
%
These neutrinos are characterised by the largest energies ever achieved in laboratory experiments,
reaching up to several TeV~\cite{Kling:2021gos}.
%
Due to the lack of dedicated instrumentation in the LHC far-forward region,
until recently these neutrinos avoided detection.
%
The recent observation of LHC neutrinos~\cite{FASER:2023zcr,SNDLHC:2023pun,CERN-FASER-CONF-2023-002} by the
  FASER~\cite{FASER:2019dxq,FASER:2022hcn} and SND@LHC~\cite{SHiP:2020sos,SNDLHC:2022ihg} far-forward experiments
demonstrates that this hitherto discarded beam can now be deployed for physics studies.
%
Beyond the ongoing Run III, a dedicated suite of upgraded far-forward
neutrino experiments would be hosted by the proposed
Forward Physics Facility (FPF)~\cite{Anchordoqui:2021ghd,Feng:2022inv} operating
concurrently with the High-Luminosity LHC~\cite{Azzi:2019yne,Cepeda:2019klc}.
%
Current and future LHC neutrinos experiments enable unprecedented scientific opportunities
for particle and astroparticle physics both within the Standard Model and beyond it,
as summarised in~\cite{Anchordoqui:2021ghd,Feng:2022inv}
and references therein.

As is well known~\cite{Conrad:1997ne,Mangano:2001mj},
measurements of neutrino structure functions~\cite{Candido:2023utz} in deep-inelastic scattering
(DIS) are sensitive probes of the parton distributions (PDFs)
of nucleons and nuclei~\cite{Ethier:2020way,Gao:2017yyd,Kovarik:2019xvh},  in particular
concerning (anti)quark  flavour separation and
strangeness~\cite{NuTeV:2007uwm,CCFR:1994ikl,Faura:2020oom,Alekhin:2014sya}.
%
Constraints arising from charged-current neutrino scattering provide information on
different flavour combinations as compared to neutral-current charged-lepton DIS,
hence being fully complementary.
%
Several  experiments have measured neutrino
DIS structure functions over a wide range of energies, and neutrino data
from CHORUS~\cite{CHORUS:2005cpn}, NuTeV~\cite{NuTeV:2001dfo},
CCFR~\cite{Yang:2000ju}, NOMAD~\cite{NOMAD:2013hbk}, and CDHS~\cite{Berge:1989hr}
and other experiments is routinely  included in global
determinations of proton~\cite{NNPDF:2021njg,Hou:2019efy,Bailey:2020ooq} and nuclear
PDFs~\cite{Eskola:2021nhw,AbdulKhalek:2022fyi,Muzakka:2022wey}.

As compared to  previous experiments,  neutrino
scattering at the LHC involves energies of up to a factor 10  higher.
%
This key feature, together with the large event rates,
up to one million muon neutrinos at the FPF~\cite{Anchordoqui:2021ghd,Feng:2022inv},
suggests that the LHC  neutrino beam may be utilized
as a precise probe of the hadron structure.
%
Initial estimates~\cite{Feng:2022inv} indicate that an extension of the coverage of
available neutrino data by an order of magnitude both at small-$x$
and large-$Q^2$ may be possible.
%
However, quantitative projections for the kinematic coverage
and experimental accuracy expected at current
and future LHC neutrino experiments are not available.
%
The lack of these projections has prevented detailed studies assessing the impact
of LHC neutrino data in global analyses of proton and nuclear PDFs, comparable to
those performed for the HL-LHC~\cite{AbdulKhalek:2018rok,Azzi:2019yne}, the Electron-Ion Collider (EIC)~\cite{AbdulKhalek:2021gbh,Khalek:2021ulf,AbdulKhalek:2019mzd}, and the
Large Hadron-electron Collider (LHeC)~\cite{AbdulKhalek:2019mps,LHeC:2020van,LHeCStudyGroup:2012zhm}. 

Here we bridge this gap by quantifying
the expected impact of  neutrino DIS structure functions at the LHC on proton and nuclear PDFs.
%
To this end, we produce simulations for  FASER$\nu$ and SND@LHC at Run III 
as well as for the proposed FPF experiments~\cite{Anchordoqui:2021ghd,Feng:2022inv,Batell:2021blf,Batell:2021aja}, FLArE, AdvSND, and FASER$\nu$2.
%
For each experiment, we determine the expected event yields in bins of $(x,Q^2,E_\nu)$
satisfying acceptance and selection cuts,
generate pseudo-data for  inclusive and charm 
structure functions, 
and estimate their dominant systematic uncertainties.
%
Subsequently, we study their impact on the proton and nuclear PDFs by means of both the Hessian profiling~\cite{Paukkunen:2014zia,  Schmidt:2018hvu, AbdulKhalek:2018rok, HERAFitterdevelopersTeam:2015cre}
of  PDF4LHC21~\cite{PDF4LHCWorkingGroup:2022cjn} (for protons) and EPPS21~\cite{Eskola:2021nhw}
(for tungsten nuclei)
within the {\sc\small xFitter}~\cite{Alekhin:2014irh, Bertone:2017tig, xFitter:2022zjb, xFitter:web} open-source QCD analysis framework,
as well as with the direct inclusion in the open-source NNPDF4.0 fitting framework~\cite{NNPDF:2021uiq}.

Our analysis reveals that  LHC neutrino structure functions can provide  stringent constraints
on the light quark and antiquark PDFs, especially on the up and down
valence quarks and on strangeness, as compared to state-of-the-art global analyses.
%
We also  find that accounting for the main systematic uncertainties does not significantly
degrade the sensitivity achieved by baseline fits considering only statistical errors.
%
We quantify the impact in our results of charm-tagged structure functions (large), study the relevance
of final-state lepton-charge identification capabilities (moderate), and compare the constraints
provided by different experiments (finding that the overall sensitivity is dominated by FASER$\nu$2).
%
We also study the implications of the resulting improvement in PDF precision
on core processes at the HL-LHC, finding a theory error reduction
of up to a factor two for selected  Higgs and gauge boson production
cross-sections. 

Our results demonstrate that the availability of far-forward neutrino detection
at the LHC effectively
provides CERN with a charged-current counterpart of the EIC,
with similar kinematic reach and complementary sensitivity on hadronic
structure.
%
Therefore, LHC neutrino experiments realise, upon Lorentz-boosting, the analog of
a ``Neutrino-Ion Collider'' at CERN
without the need of new accelerator infrastructure or additional energy consumption.

The outline of this paper is as follows.
%
Sect.~\ref{sec:dis_pseudodata} discusses the procedure
adopted to generate projections for neutrino DIS structure functions at the LHC
and the methodology used to include these into PDF fits.
%
The impact of such LHC structure function measurements on proton and nuclear
PDFs is quantified in Sect.~\ref{sec:protonPDFs}, with the
associated implications for precision phenomenology
at the HL-LHC assessed in Sect.~\ref{sec:pheno}.
%
We summarise in Sect.~\ref{sec:summary}, where we also consider possible
directions for follow-up research.
%
Additional results are collected in two appendices:
App.~\ref{app:fasernu_runIII_impact} studies the constraints provided by
FASER$\nu$ measurements at Run III, and App.~\ref{app:nPDF_impact_appendix}
quantifies the stability of the nPDF impact projections with respect
to input variations.

Together with this paper,
we release the generated  DIS structure function pseudo-data for the LHC
neutrino experiments, which may be helpful
to inform related studies of high-energy
neutrino production and scattering at the LHC.



