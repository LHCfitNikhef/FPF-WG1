\clearpage
\section{Summary and outlook}
\label{sec:summary}

In this work, we have quantified the impact of ongoing and proposed
experiments detecting the scattering of energetic neutrinos produced
by LHC collisions in the forward direction.
%
By means of estimating the expected deep-inelastic scattering
event rates differential
in $x$, $Q^2$, and $E_\nu$ for these various experiments, and
evaluating the associated theoretical predictions for
both inclusive and charm production processes, we have
assessed the expected reduction of PDF uncertainties with respect
to modern fits of proton and nuclear structure.
%
To this end, we have studied the robustness of the results
with respect the modelling of systematic uncertainties, the role of lepton
charge separation, and the impact of the selection and acceptance cuts.

We demonstrate that neutrino DIS measurements at the proposed FPF
experiments would have a significant potential to improve our current
knowledge of quark and antiquark flavour separation in protons
and heavy nuclei, including strangeness,
and that the FASER$\nu$ and SND@LHC
experiments may provide the first-ever
constraints on hadron structure from collider neutrinos.
%
Together with this paper, we also
release the pseudodata and the corresponding theory
calculations produced in this work.
%
These should be of relevance for colleagues interested in the physics
of forward
neutrino scattering at the LHC, not only for hadronic structure studies
but also for neutrino (effective)
interactions and BSM searches e.g. of sterile neutrino oscillations.

Several avenues extending the results of this work may be foreseen.
%
First of all, as the design of the proposed FPF experiments
becomes more advanced and concrete, it should be possible
to construct more accurate estimates of the systematic
uncertainties and of their bin-by-bin correlations,
which may eventually become the limiting factor
for hadron structure analyses with LHC neutrinos.
%
Second, scattering analyses with LHC neutrinos would benefit from
the use of Monte Carlo event generators accounting
for higher-order QCD corrections.
%
As compared to the currently used LO generators,
these would improve the modelling of the final-state leptonic
and hadronic kinematics, which in turn determine the acceptance rate
of the reconstructed events.
%
Further, the predictions from such precise Monte Carlo generators
should eventually be folded with a full-fledged detector simulation
in order to robust determine selection efficiencies, e.g.
such as those related to charm tagging.

Third, while in this work we have focused on the deep-inelastic
region, ongoing and future LHC experiments also provide
important information on the shallow-inelastic scattering (SIS)
region at low values of $Q$~\cite{Candido:2023utz}, relevant for inclusive cross-sections
entering atmospheric and oscillation neutrino experiments.
%
By means of a similar approach, it should be possible to quantify
the improvements that LHC neutrino data can provide on models
of neutrino scattering in the poorly-understood SIS region.
%
Finally, in this work we have taken the incoming muon neutrino flux
as a given and neglected any associated uncertainties.
%
However, ultimately one may need to constrain at the same time
the incoming fluxes and the neutrino scattering cross-sections
from the recorded event rates.
%
Such joint interpretation would require extending the present
analysis with additional theory nuisance parameters modelling
deviations of the neutrino flux with respect to the baseline predictions.

All in all, our findings highlight how exploiting
the LHC neutrino beam for hadron structure studies effectively
extends the LHC with a ``Neutrino-Ion Collider'', a charged
current-counterpart of the EIC, and that we are still
scratching the surface of the uncharted
scientific potential of such experiments.

\subsection*{Acknowledgments}
%
We are grateful to many colleagues involved in the Forward
Physics Facility initiative for many illuminating
discussions and encouragement.
%
We thank Akitaka Ariga and Tomoko Ariga for information
concerning FASER$\nu$2,
Wenjie Wu and Steven Linden concerning FLArE,
and Antonia di Crescenzo concerning AdvSND.
