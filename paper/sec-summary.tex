\clearpage
\section{Summary and outlook}
\label{sec:summary}

In this work, we have quantified the impact of ongoing and proposed
experiments detecting the scattering of energetic neutrinos produced
by LHC collisions in the forward direction.
%
By means of estimating the expected deep-inelastic scattering
event rates differential
in $x$, $Q^2$, and $E_\nu$ for these various experiments, and
evaluating the associated theoretical predictions for
both inclusive and charm production processes, we have
assessed the expected reduction of PDF uncertainties with respect
to modern fits of proton and nuclear structure.
%
To this end, we have studied the robustness of the results
with respect the modelling of systematic uncertainties, the role of lepton
charge separation, and the impact of the selection and acceptance cuts.

We demonstrate that neutrino DIS measurements at the proposed FPF
experiments would have a significant potential to improve our current
knowledge of quark and antiquark flavour separation in protons
and heavy nuclei, including strangeness,
and that the FASER$\nu$ and SND@LHC
experiments may provide the first-ever
constraints on hadron structure from collider neutrinos.
%
Together with this paper, we also
release the pseudodata and the corresponding theory
calculations produced in this work.
%
These should be of relevance for colleagues interested in the physics
of forward
neutrino scattering at the LHC, not only for hadronic structure studies
but also for neutrino (effective)
interactions and BSM searches e.g. of sterile neutrino oscillations.

Several avenues extending the results of this work may be foreseen.
%
First of all, as the design of the proposed FPF experiments
becomes more advanced and concrete, it should be possible more reliable
estimate of corrections, may become the limiting factor
for this kind of analyses.
%
Improve event generation, based on {\sc\small Pythia}
modern QCD generator with higher order QCD corrections,
why improve estimates better modelling of final state acceptance.
%
Analysis also SIS region low $Q$, improve modelling of low-$Q$ cross-sections
relevant for atmospheric and oscillation neutrino experiments.
%
Neglect flux uncertainties, joint determination

All in all, Our analysis highlights how exploiting the potential
 of this unique neutrino beam  effectively
extends the LHC with a ``Neutrino-Ion Collider''.



\subsection*{Acknowledgments}
%
We are grateful to many colleagues involved in the Forward
Physics Facility initiative for many illuminating
discussions and encouragement.
