\section{Summary and outlook}
\label{sec:summary}

In this work we have, for the first time,
quantified the impact that measurements of high-energy
neutrino DIS structure functions at the LHC would have on the quark
and gluon structure of the nucleon.
%
Our analysis has involved the generation of
DIS pseudo-data fully differential in $x$, $Q^2$, and $E_\nu$
for the various ongoing and proposed far-forward
neutrino LHC experiments, including an initial estimate
of their associated systematic uncertainties.
%
Consistent results are obtained from both the Hessian profiling
of PDF4LHC21 and from the direct inclusion of LHC neutrino
structure functions in the NNPDF4.0 global
fit, revealing a reduction of PDF uncertainties in the
light quark sector, in particular concerning strangeness, as well
as for the large-$x$ charm PDF in the case of NNPDF4.0.
%
We have then assessed the  robustness of the results upon
removing charm-tagged data and final-state lepton-charge
identification, as well as upon the combination
of all FPF experiments within a single analysis.

We have also demonstrated the rich interplay between
far-forward and central measurements
at the HL-LHC, by providing predictions
for a range of Higgs and gauge boson production processes,
both for integrated cross-sections in the fiducial region and for
single-differential distributions.
%
This HL-LHC phenomenological analysis suggests that a reduction
of the  PDF uncertainty by up to a factor two
may be within reach for some of these cross-sections,
in the most optimistic scenario.
%
As was the case at the PDF level, also for 
the HL-LHC projections results based on  PDF4LHC21 and NNPDF4.0
are in qualitative agreement.

Several avenues extending the results of this work may be foreseen.
%
First of all, as the design of the proposed FPF experiments
 advances, it will be possible to derive
 more accurate estimates of the systematic
uncertainties (and of their correlations) 
which eventually become the limiting factor.
%
This will allow studying whether  improved detection methods,
novel reconstruction techniques i.e. based
on deep learning, or combining information from different
experiments can  push down the systematic uncertainties
affecting the measurements.
%
Second, the modelling of neutrino scattering at the LHC would benefit from
the use of Monte Carlo event generators accounting
for higher-order QCD corrections.
%
As compared to the currently used LO generators,
these would improve the description of the final-state
kinematics, which in turn determine the acceptance rate
of the reconstructed events.
%
Furthermore, the predictions from such precise Monte Carlo generators
should eventually be folded with a full-fledged detector simulation
in order to robustly determine selection efficiencies, e.g.
such as those related to charm and $D$-meson tagging.

Third, while here we focus on the DIS
region, ongoing and future LHC neutrino experiments also provide
important information on shallow-inelastic scattering (SIS)
 at lower values of $Q$~\cite{Jeong:2023hwe,Candido:2023utz},
 which in turn are relevant for inclusive cross-sections
entering atmospheric and oscillation neutrino experiments.
%
By following the approach presented in this work, it should be possible to quantify
the improvements that LHC data provides on models
of neutrino scattering in this poorly-understood SIS region.
%
Finally, in this work we have taken the incoming neutrino fluxes
as an external input and neglected any associated uncertainties.
%
However, measurements of these fluxes provides
unique information on light and heavy forward hadron
production in QCD, and in particular open
a new window to the small-$x$ gluon PDF.
%
For this reason, ultimately one needs to simultaneously constrain 
the incoming fluxes and the neutrino scattering cross-sections
from the measured event rates.
%
Such joint interpretation would require extending the present
analysis with a data-driven parametrisation of the neutrino fluxes,
which could subsequently be compared with different theoretical predictions.

Our findings highlight how exploiting the LHC neutrino beam for hadron structure
studies effectively provides CERN with a ``Neutrino-Ion Collider'', a charged
current-counterpart of the EIC, without  changes in
its accelerator infrastructure or additional energy costs.
%
In addition for their intrinsic interest for hadronic science,
measurements of neutrino structure function at the LHC
provide a novel, and until now ignored, handle to inform theoretical
predictions of hard-scattering cross-sections  at the HL-LHC.

\begin{center}
\rule{5cm}{.1pt}
\end{center}
\bigskip

Together with this paper, we 
make public in Zenodo
the corresponding {\sc\small LHAPDF} grids~\cite{Buckley:2014ana}
for the PDF4LHC21 and NNPDF4.0 fits including FPF data.
%
We also release
the generated LHC neutrino DIS structure function pseudo-data
and the corresponding theory
calculations, for all scenarios considered in this work.
%
These projections for neutrino structure functions
should be of relevance for a broad range of
applications related to forward
neutrino scattering at the LHC, from tests
of lepton flavour university at the TeV scale
in the neutrino sector to probes of anomalous neutrino
interactions and searches for sterile neutrinos
distorting  oscillation patterns.

\subsection*{Acknowledgments}
%
We are grateful to many colleagues involved in the Forward
Physics Facility initiative for illuminating
discussions and encouragement along the course of this project,
in particular Jamie Boyd, Jonathan Feng, and Albert de Roeck.
%
We thank specially Felix Kling for many useful discussions
and for providing updated predictions for the neutrino fluxes.
%
We thank Akitaka Ariga and Tomoko Ariga for discussion
concerning FASER$\nu$2, Milind Diwan,
Wenjie Wu, and Steven Linden concerning FLArE,
and Antonia di Crescenzo for information concerning AdvSND.


The work of M.~F. was supported by NSF Grant PHY-2210283 and was also supported by NSF Graduate Research Fellowship Award No. DGE-1839285.
%
The work of T.~G., G.~M., and J.~R. is partially supported by NWO, the Dutch Research Council.
%
The work of T.~R. and J.~R. is partially supported by an ASDI2020
Fellowship from the Netherlands eScience Center.
%
The work of T.~M is supported by the National Science Centre, Poland, research grant No. 2021/42/E/ST2/00031.
