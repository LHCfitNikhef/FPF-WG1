\clearpage
\section{Summary and outlook}
\label{sec:summary}

In this work we have, for the first time,
quantified the impact that measurements of high-energy
neutrino DIS structure functions at the LHC would have on the quark
and gluon structure of the nucleon.
%
Our analysis has required the generation of
DIS pseudo-data fully differential in $x$, $Q^2$, and $E_\nu$
for the various ongoing and proposed far-forward
neutrino LHC experiments, including an initial estimate
of their associated systematic uncertainties.
%
Consistent results are obtained from both the Hessian profiling
of PDF4LHC21 and from the direct inclusion of LHC neutrino
structure functions in the NNPDF4.0 global
fit, revealing a reduction of PDF uncertainties in the
light quark sector, in particular concerning strangeness, as well
as for the large-$x$ charm PDF in the case of NNPDF4.0.
%
We have then assessed the  robustness of the results upon
removing charm-tagged data and final-state lepton-charge
identification, as well as upon the combination
of all FPF experiments within a single analysis.

We have also demonstrated the rich interplay between
far-forward and central measurements
at the HL-LHC by providing predictions
for a range of Higgs and gauge boson production processes,
both for integrated cross-sections in the fiducial region and for
single-differential distributions.
%
This HL-LHC phenomenological analysis suggests that a reduction
of the  PDF uncertainty by up to a factor two
may be within reach for some of these cross-sections,
in the most optimistic scenario.
%
As was the case at the PDF level, also for 
the HL-LHC projections results based on  PDF4LHC21 and NNPDF4.0
are in qualitative agreement.
%
Together with this paper, we also
release the corresponding {\sc\small LHAPDF} grids~\cite{Buckley:2014ana}
for PDF4LHC21 and NNPDF4.0 including FPF data,
as well as the LHC neutrino DIS structure function pseudo-data
and the corresponding theory
calculations for all scenarios considered in this work.
%
The latter should be of relevance for colleagues interested in the physics
of forward
neutrino scattering at the LHC, not only for hadronic structure studies
but also for neutrino (effective)
interactions and BSM searches e.g. of sterile neutrino oscillations.

Several avenues extending the results of this work may be foreseen.
%
First of all, as the design of the proposed FPF experiments
becomes more advanced and concrete, it should be possible
to construct more accurate estimates of the systematic
uncertainties and of their bin-by-bin correlations,
which may eventually become the limiting factor
for hadron structure analyses with LHC neutrinos.
%
In particular, improved experiment design novel reconstruction techniques i.e. based
on deep learning or combination of information from different
experiments may be key to further reduce systematic
uncertainties as required to maximise the impact
of neutrino DIS measurements at the LHC.
%
Second, scattering analyses with LHC neutrinos would benefit from
the use of Monte Carlo event generators accounting
for higher-order QCD corrections.
%
As compared to the currently used LO generators,
these would improve the modelling of the final-state leptonic
and hadronic kinematics, which in turn determine the acceptance rate
of the reconstructed events.
%
Further, the predictions from such precise Monte Carlo generators
should eventually be folded with a full-fledged detector simulation
in order to robust determine selection efficiencies, e.g.
such as those related to charm tagging.

Third, while in this work we have focused on the deep-inelastic
region, ongoing and future LHC experiments also provide
important information on the shallow-inelastic scattering (SIS)
region at low values of $Q$~\cite{Jeong:2023hwe,Candido:2023utz}, relevant for inclusive cross-sections
entering atmospheric and oscillation neutrino experiments.
%
By means of a similar approach, it should be possible to quantify
the improvements that LHC neutrino data can provide on models
of neutrino scattering in the poorly-understood SIS region.
%
Finally, in this work we have taken the incoming muon neutrino flux
as a given and neglected any associated uncertainties.
%
small-x QCD, gluon PDF
%
However, ultimately one may need to constrain at the same time
the incoming fluxes and the neutrino scattering cross-sections
from the recorded event rates.
%
Such joint interpretation would require extending the present
analysis with additional theory nuisance parameters modelling
deviations of the neutrino flux with respect to the baseline predictions.

All in all, our findings highlight how exploiting
the LHC neutrino beam for hadron structure studies effectively
extends the LHC with a ``Neutrino-Ion Collider'', a charged
current-counterpart of the EIC, and that we are still
scratching the surface of the uncharted
scientific potential of such experiments.

\subsection*{Acknowledgments}
%
We are grateful to many colleagues involved in the Forward
Physics Facility initiative for illuminating
discussions and encouragement along the course of this project,
in particular Jamie Boyd, Jonathan Feng, and Albert de Roeck.
%
We thank specially Felix Kling for many useful discussions
and for providing updated predictions for the neutrino fluxes.
%
We thank Akitaka Ariga and Tomoko Ariga for discussion
concerning FASER$\nu$2, Milind Diwan,
Wenjie Wu, and Steven Linden concerning FLArE,
and Antonia di Crescenzo for information concerning AdvSND.


The work of M.~F. was supported by NSF Grant PHY-2210283 and was also supported by NSF Graduate Research Fellowship Award No. DGE-1839285.
%
The work of T.~G., G.~M., and J.~R. is partially supported by NWO, the Dutch Research Council.
%
The work of T.~R. and J.~R. is partially supported by an ASDI2020
Fellowship from the Netherlands eScience Center.
