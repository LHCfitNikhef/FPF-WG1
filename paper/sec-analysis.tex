\clearpage
\section{Analysis settings}
\label{sec:settings}

Once we have generated pseudo-data for the 
neutrino DIS charged-current structure functions
in bins of $(x,Q^2,y)$, and assuming that uncertainties
are purely of statistical origin, we can assess their
impact on proton PDFs.
%
At a first step we neglect nuclear effects
and treat data as corresponding to a free isoscalar
nucleon with $A=1$.

The expected impact of the FPF data on global PDF fits is assessed by a profiling procedure~\cite{Paukkunen:2014zia, Schmidt:2018hvu, AbdulKhalek:2018rok, HERAFitterdevelopersTeam:2015cre}, based on minimizing the function
%TODO correlations are also included here, as we hopefully get there in the end -- if not, remove terms involving correlations+pseudodata
\begin{equation}
\chi^2 = 
\sum_{i=1}^{N_{\textrm{bins}}} 
\frac{\left(  \sigma_i^{\textrm{pd}}
            + \Gamma_i^{\alpha,\textrm{pd}}
              b_\alpha^{\textrm{pd}}
            - \sigma_i^{\textrm{th}}
            - \Gamma_i^{\beta,\textrm{th}}
              b_\beta^{\textrm{th}}
     \right)^2
     }{\Delta_i^2}
+ \sum_\alpha (b_\alpha^{\textrm{pd}})^2
+ \sum_\beta  (b_\beta^{\textrm{th}})^2 \, .
\label{eq:profilingchi2}
\end{equation}
Here the pseudodata 
$\sigma_i^{\textrm{pd}}$ 
is obtained from the central theoretical prediction 
$\sigma_i^{\textrm{th}}$ 
for each bin $i$ out of $N_{\textrm{bins}}$ by varying it within bounds obtained from the statistical (systematic) uncertainties 
$\delta_i^{\textrm{stat}}$ ($\delta_i^{\textrm{syst}}$), 
obtained in the procedure described in Section~\ref{sec:pseudo-data_generation}, 
as 

\begin{equation}
\sigma_i^{\textrm{pd}}
=
\sigma_i^{\textrm{th}}
\left( 1 + r_i \sqrt{     (\delta_i^{\textrm{stat}})^2
                      + (f \delta_i^{\textrm{syst}})^2 }
\right).
\end{equation}
Here, $r_i$ is a univariate Gaussian random number and $f = 0.5$ an effective correction factor 
accounting for the improved constraining power of the same pseudodata set using correlated systematics, compared to solely adding all sources of uncertainty in quadrature.
%
The correlated uncertainties for the pseudodata and the theoretical prediction 
are contained in the nuisance parameter vectors $b^{\textrm{pd}}$ and $b^{\textrm{th}}$, respectively, and the uncorrelated uncertainties in $\Delta_i$.
%
Their effect on $\sigma^{\textrm{th}}$ and $\sigma^{\textrm{pd}}$
is described by the matrices $\Gamma_i^{\textrm{pd}}$ and $\Gamma_i^{\textrm{th}}$.
The indices $\alpha$ and $\beta$ then run over the uncertainty nuisance parameters for the pseudodata and the theoretical prediction, respectively.
%
The nuisance parameter values $b^{\textrm{th(min)}}$ that minimize Eq.~\eqref{eq:profilingchi2} give the central PDFs $f'_0$ optimized to the profiled dataset in the form
\begin{equation}
f_0' = f_0
      + \sum_\beta b_\beta^{\textrm{th(min)}} 
        \left(  \frac{f_\beta^+   -  f_\beta^- }{2}
              -    b_\beta^{\textrm{th(min)}}
                \frac{f_\beta^+ + f_\beta^- - 2f_0}{2}
        \right),
\end{equation}
where $f_0$ is the original central PDF and the up and down variation eigenvectors are given by $f^+, f^-$. The decrease in the uncertainties of the shifted PDFs indicates the enhancement that including the FPF data in the global PDF fit could bring. Here, the profiling studies are performed using version 2.2.1 of the \textsc{xFitter} open-source QCD analysis framework~\cite{Alekhin:2014irh, Bertone:2017tig, xFitter:2022zjb, xFitter:web}.
%


\subsection{Modelling DIS structure functions}

 %
Theoretical predictions for neutrino DIS structure functions are evaluated with the {\sc\small EKO}~\cite{Candido:2022tld}
and {\sc\small YADISM}~\cite{yadism,Candido:2023utz} programs interfaced to {\sc\small PineAPPL}~\cite{Carrazza:2020gss, christopher_schwan_2023_7995675}, with the latter providing the fast interpolation grids to be used in the fit.
%
To this end, a new interface between  {\sc\small PineAPPL} and {\sc\small xFitter} has been developed and is available with the open-source {\sc\small xFitter} code.
