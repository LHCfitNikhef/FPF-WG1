\clearpage
\section{Implications for HL-LHC processes}
\label{sec:pheno}

The reduction of PDF uncertainties enabled by the neutrino DIS measurements
at the FPF presented in Sect.~\ref{sec:protonPDFs}
makes possible more precise theoretical predictions for key processes at the
HL-LHC.
%
Here we present an initial study of the phenomenological implications
of PDFs enhanced with LHC neutrino data.
%
We adopt the same settings as in the LHC phenomenology analysis presented
in the PDF4LHC21 combination paper~\cite{PDF4LHCWorkingGroup:2022cjn} and provide predictions
both for inclusive fiducial cross-sections and for differential distributions.
%
Specifically, we present results for 
single and double gauge boson production, inclusive top quark pair production,
and Higgs production in gluon
fusion and in association with a vector boson.
%
We evaluate these cross-sections using NLO matrix elements
which include  both in the
QCD and electroweak corrections using
{\sc\small mg5\_amc@nlo}~\cite{Frederix:2018nkq}
interfaced to {\sc\small PineAPPL}~\cite{Carrazza:2020gss}.
%
For all processes, realistic selection and acceptance cuts on the final state particles
have been applied, and PDF uncertainty bands correspond to 90\% CL
uncertainties.
%
No further theory uncertainties are considered in this
analysis.

Fig.~\ref{fig:NNPDF40_pheno_integrated} displays
fiducial cross-section for representative LHC processes at $\sqrt{s}=14$ TeV
evaluated with NNPDF4.0 NNLO, compared with the fits including the FPF structure function projections.
%
For the latter, we display the variants based only on statistical uncertainties and that
which includes also systematic errors.
%
The central values are set to be the same as in the original NNPDF4.0 calculation in all cases.
%
See~\cite{NNPDF:2021njg} for the calculational settings.
%
From top to bottom, we show inclusive Drell-Yan production ($Z, W^+, W^-$), Higgs production
in vector-boson fusion, Higgs associated
production, and diboson production ($W^+W^-$, $W^+Z$, $W^-Z$).
%
We focus on processes dominated by quark-quark and quark-antiquark scattering, given
that the FPF structure functions would not have any impact on gluon-initiated
processes such as top quark pair production or Higgs production in gluon fusion.


%%%%%%%%%%%%%%%%%%%%%%%%%%%%%%%%%%%%%%%%%%%%%%%%%%%%%%%%%%%%%%%%%%%%%%%%
\begin{figure}[t]
\centering
\includegraphics[width=0.32\textwidth]{plots/LHCpheno/NNPDF_DY_14TEV_40_PHENO-integrated.pdf}
\includegraphics[width=0.32\textwidth]{plots/LHCpheno/NNPDF_WM_14TEV_40_PHENO-integrated.pdf}
\includegraphics[width=0.32\textwidth]{plots/LHCpheno/NNPDF_WP_14TEV_40_PHENO-integrated.pdf}
\includegraphics[width=0.32\textwidth]{plots/LHCpheno/NNPDF_HVBF_14TEV_40_PHENO-integrated.pdf}
\includegraphics[width=0.32\textwidth]{plots/LHCpheno/NNPDF_HWP_14TEV_40_PHENO-integrated.pdf}
\includegraphics[width=0.32\textwidth]{plots/LHCpheno/NNPDF_HWM_14TEV_40_PHENO-integrated.pdf}
\includegraphics[width=0.32\textwidth]{plots/LHCpheno/NNPDF_WPWM_14TEV_40_PHENO-integrated.pdf}
\includegraphics[width=0.32\textwidth]{plots/LHCpheno/NNPDF_WPZ_14TEV_40_PHENO-integrated.pdf}
\includegraphics[width=0.32\textwidth]{plots/LHCpheno/NNPDF_WMZ_14TEV_40_PHENO-integrated.pdf}
\caption{Fiducial cross-section for representative LHC processes at $\sqrt{s}=14$ TeV
evaluated with NNPDF4.0 NNLO, compared with the fits including the FPF structure function projections.
%
For the latter, we display the variants based only on statistical uncertainties and that
which includes also systematic errors.
%
The central values are set to be the same as in the original NNPDF4.0 calculation in all cases.
%
See~\cite{NNPDF:2021njg} for the calculational settings.
%
From top to bottom, we show inclusive Drell-Yan production ($Z, W^+, W^-$), Higgs production
in vector-boson fusion, Higgs associated
production, and diboson production ($W^+W^-$, $W^+Z$, $W^-Z$).
%
}
\label{fig:NNPDF40_pheno_integrated}
\end{figure}
%%%%%%%%%%%%%%%%%%%%%%%%%%%%%%%%%%%%%%%%%%%%%%%%%%%%%%%%%%%%%%%%%%%%%%%%

The corresponding comparison at the level of differential distributions
are shown in Fig.~\ref{fig:NNPDF40_pheno_differential}.
%
As done for the fiducial cross-sections of Fig.~\ref{fig:NNPDF40_pheno_integrated},
we only indicate the relative PDF uncertainty in each fit, with central values
assumed to be the same by construction.

%%%%%%%%%%%%%%%%%%%%%%%%%%%%%%%%%%%%%%%%%%%%%%%%%%%%%%%%%%%%%%%%%%%%%%%%
\begin{figure}[t]
\centering
\includegraphics[width=0.49\textwidth]{plots/LHCpheno/NNPDF_DY_14TEV_40_PHENO-global.pdf}
\includegraphics[width=0.49\textwidth]{plots/LHCpheno/NNPDF_WM_14TEV_40_PHENO-global.pdf}
\includegraphics[width=0.49\textwidth]{plots/LHCpheno/NNPDF_WP_14TEV_40_PHENO-global.pdf}
\includegraphics[width=0.49\textwidth]{plots/LHCpheno/NNPDF_HVBF_14TEV_40_PHENO-global.pdf}
\includegraphics[width=0.49\textwidth]{plots/LHCpheno/NNPDF_HWP_14TEV_40_PHENO-global.pdf}
\includegraphics[width=0.49\textwidth]{plots/LHCpheno/NNPDF_HWM_14TEV_40_PHENO-global.pdf}
\includegraphics[width=0.49\textwidth]{plots/LHCpheno/NNPDF_WPWM_14TEV_40_PHENO-global.pdf}
\includegraphics[width=0.49\textwidth]{plots/LHCpheno/NNPDF_WPZ_14TEV_40_PHENO-global.pdf}
%\includegraphics[width=0.49\textwidth]{plots/LHCpheno/NNPDF_WMZ_14TEV_40_PHENO-global.pdf}
\caption{Same as Fig.~\ref{fig:NNPDF40_pheno_integrated}
for the corresponding differential distributions.
%
}
\label{fig:NNPDF40_pheno_differential}
\end{figure}
%%%%%%%%%%%%%%%%%%%%%%%%%%%%%%%%%%%%%%%%%%%%%%%%%%%%%%%%%%%%%%%%%%%%%%%%
