%%%%%%%%%%%%%%%%%%%%%%%%%%%%%%%%%%%%%%%%%%%
%-----------------------------------------------------------------
\begin{table}[t]
  \centering
  \small
  \renewcommand{\arraystretch}{1.70}
\begin{tabularx}{\textwidth}{X|c|c|c|c|c|c}
\toprule
Detector & $\quad$ $N_{\nu_e}$ $\quad$ &$\quad$ $N_{\bar{\nu}_e}$$\quad$   &   $N_{\nu_e} + N_{\bar{\nu}_e}$ &
$\quad$$N_{\nu_\mu}$ $\quad$ & $\quad$ $N_{\bar{\nu}_\mu}$ $\quad$  &   $N_{\nu_\mu} + N_{\bar{\nu}_\mu}$ \\
\midrule
\midrule
FASER$\nu$  &    &    &   &  1695.1  &  482.6  &  2177.6  \\
SND@LHC  &    &    &   &  452.1 & 161.2   &  613.3  \\
\midrule
\midrule
FASER$\nu$2  &    &    &   & 114652.8  & 27194.0    & 141492.7   \\
AdvSND~(far)  &    &    &   &   &    &    \\
FLArE10 &   15139.2 & 7946.2   &  22437.7 &   &    &    \\
  \bottomrule
\end{tabularx}
\vspace{0.2cm}
\caption{\small Integrated event yields for the five detectors considered,
  separated into electron neutrinos and antineutrinos,
  muon neutrinos and antineutrinos, and their sum.
  %
  These event yields are computed from Eq.~(\ref{eq:event_yields})
  imposing DIS kinematics, $Q^2 \ge 2$ GeV$^2$ and $W^2 \ge 4$ GeV$^2$.
  %
  Whenever available, the
  numbers in parenthesis indicate the event rates corresponding to charm
  production.
  \label{tab:integrated_rates}
}
\end{table}
%-----------------------------------------------------------------
%%%%%%%%%%%%%%%%%%%%%%%%%%%%%%%%%%%%%%%%%%%%
