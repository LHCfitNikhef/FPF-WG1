%%%%%%%%%%%%%%%%%%%%%%%%%%%%%%%%%%%%%%%%%%%
%-----------------------------------------------------------------
\begin{table}[t]
  \centering
  \small
  \renewcommand{\arraystretch}{1.70}
  \begin{tabularx}{0.8\textwidth}{p{0.3\textwidth}|c|c|c}
    \toprule
    Detector & Before Cuts & DIS Cut & Acceptance Cut \\
    \midrule
    \midrule
    FASER$\nu$ & 1.2k , 4.1k & 1.2k , 4.1k & 0.61k , 1.8k \\
    SND@LHC & 0.28k , 0.86k & 0.28k , 0.86k & 0.25k , 0.70k \\
    \midrule
    \midrule
    FASER$\nu$2 & 270k , 980k & 270k , 980k  & 172k , 510k \\
    AdvSND-far & 19k , 66k & 19k , 66k & 17k , 56k\\
    FLArE10 & 65k , 202k & 65k , 202k  & 64k , 114k \\
    FLArE100 & 427k , 1.3M &427k , 1.3M  & 420k , 680k \\
    \bottomrule
  \end{tabularx}
  \vspace{0.2cm}
\caption{Cut table showing the impact on the event yield after employing DIS cuts, $W^2 > 4{\rm GeV}^2$ and $Q^2 > 2{\rm GeV}^2$, and experimental acceptances from Table~\ref{tab:FPF_experiments}. The first number for each entry is the numer of $\nu_e + \bar{\nu}_e$ events and the second is for $\nu_{\mu} + \bar{\nu}_{\mu}$.  For the region of parameter space that we study, DIS cuts remove very few events ($\lesssim 1\%$), while acceptance cuts depend on the detector but can remove as many as 50$\%$.}
\label{tab:acceptance}
\end{table}
%-----------------------------------------------------------------
%%%%%%%%%%%%%%%%%%%%%%%%%%%%%%%%%%%%%%%%%%%%
