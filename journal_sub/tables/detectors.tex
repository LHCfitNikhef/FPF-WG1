%-----------------------------------------------------------------
\begin{table*}[htbp]
  \centering
  \footnotesize
  \renewcommand{\arraystretch}{1.50}
\begin{tabularx}{\textwidth}{Xccccc}
\toprule
Detector &  Rapidity &  Target & Charge ID & Acceptance  & Performance \\
\midrule
\midrule
\multirow{3}{*}{FASER$\nu$}  &  \multirow{3}{*}{ $\eta_\nu \ge 8.5$}  &   \multirow{2}{*}{Tungsten}  & \multirow{3}{*}{muons}      &   $E_\ell,E_h \gsim 100$ GeV   &      $\delta E_\ell \sim 30\% $    \\
&   &   \multirow{2}{*}{(1.1 tonnes)}  &       &  $\tan \theta_\ell \lsim 0.025 $ (charge ID)   &
$\delta \theta_\ell \sim 0.06$ mrad        \\
&   &     &       &  reco $E_h$ \& charm ID   &      $\delta E_h \sim 30\%$     \\
\midrule
\multirow{2}{*}{SND@LHC}  & \multirow{2}{*}{ $7.2 \le \eta_\nu \le 8.4$}   &  Tungsten   &   \multirow{2}{*}{n/a}    &  $E_\ell,E_h \gsim 20 $ GeV     &    \multirow{2}{*}{n/a}    \\
  &    &  (0.83 tonnes)   &  &  $\theta_\mu \lsim 0.15, \theta_e \lsim 0.5$         &       \\
\midrule
\midrule
\multirow{3}{*}{FASER$\nu$2}  & \multirow{3}{*}{ $\eta_\nu \ge 8.5$}  & \multirow{2}{*}{Tungsten}    &   \multirow{3}{*}{muons}     &   $E_\ell,E_h \gsim 100$ GeV  &    $\delta E_\ell \sim 30\% $     \\
  &   &  \multirow{2}{*}{(20 tonnes)}   &       &  $\tan \theta_\ell \lsim 0.05$ (charge ID)  &   $\delta \theta_\ell \sim 0.06$ mrad      \\
  &   &     &       &  reco $E_h$ \& charm ID   &  $\delta E_h \sim 30\%$        \\
\midrule
\multirow{3}{*}{AdvSND-far}  &   \multirow{3}{*}{ $7.2 \le \eta_\nu \le 8.4$}  &
\multirow{2}{*}{Tungsten}   &   \multirow{3}{*}{muons}    &  $E_\ell,E_h \gsim 20 $ GeV  & \multirow{3}{*}{n/a}          \\
  &   &   \multirow{2}{*}{(5 tonnes)}  &        & $\theta_\mu \lsim 0.15, \theta_e \lsim 0.5$     &           \\
  &   &     &       &  reco $E_h$   &           \\
\midrule
\multirow{3}{*}{FLArE ({\bf *})}  & \multirow{3}{*}{$\eta_\nu \ge 7.5$} & \multirow{2}{*}{LAr}  & \multirow{3}{*}{muons}  &  $E_\ell,E_h \gsim 2$ GeV, $E_e \lsim 2$ TeV    &    $\delta E_e \sim 5\% $,  $\delta E_\mu \sim 30\% $ \\
&   &  \multirow{2}{*}{(10,~100~tonnes)}   &   & $\theta_\mu \lsim 0.025$, $\theta_e \lsim 0.5$ &    $\delta \theta_\ell \sim 15 $ mrad\\
 &   &     &  & reco $E_h$  &    $\delta E_h \sim 30\% $   \\
  \bottomrule
\end{tabularx}
\vspace{0.2cm}
\caption{\small For each of the far-forward LHC neutrino experiments considered,
   we indicate their neutrino pseudo-rapidity coverage, target material, whether
  they can identify the sign of the outgoing charged lepton,
  the acceptance for the charged lepton and hadronic final state,
  and the expected reconstruction performance.
  %
  We consider separately acceptance and performance for electron and muon
  neutrinos.
  %
  For FLArE, we assume that muons would be measured in the FASER2 spectrometer
  situated downstream in the FPF cavern.
  %
  See the description of each experiment in the text for more details.
  %
  For our projections we assume that FASER$\nu$ and SND@LHC acquire data
  for the Run III period ($\mathcal{L}=150$ fb$^{-1}$), while FASER$\nu$2, AdvSND, and FLArE take data
  for the complete HL-LHC period ($\mathcal{L}=3$ ab$^{-1}$).
  %
  In the case of FLArE, we consider projections for fiducial volumes corresponding to both 10 and 100 tonne detectors, which we denote for the two detectors as FLArE(*).
  \label{tab:FPF_experiments}
}
\end{table*}
%-----------------------------------------------------------------
